\documentclass[11pt, oneside]{article}      % use "amsart" instead of "article" for AMSLaTeX format
\usepackage[margin=1in]{geometry}                       % See geometry.pdf to learn the layout options. There are lots.
\geometry{a4paper}                          % ... or a4paper or a5paper or ... 
%\geometry{landscape}                       % Activate for rotated page geometry
%\usepackage[parfill]{parskip}          % Activate to begin paragraphs with an empty line rather than an indent
\usepackage{graphicx}               % Use pdf, png, jpg, or eps§ with pdflatex; use eps in DVI mode
                                % TeX will automatically convert eps --> pdf in pdflatex        
\usepackage{amssymb}
\graphicspath{ {/Users/Jiang/Desktop/2016 Fall Semester/Math 87 Math Modeling/HW 5/} }
\usepackage{amsmath}
\usepackage{enumitem}
\usepackage{todonotes}
\usepackage{listings}
\usepackage{courier}
\usepackage{color}
\usepackage{titling}
\setlength{\droptitle}{-20pt}
\renewcommand{\floatpagefraction}{.8}%
\renewcommand{\topfraction}{.9}

%\usepackage[left =1cm,right=1cm,top=1cm,bottom=2cm]{geometry}

\definecolor{dkgreen}{rgb}{0,0.6,0}
\definecolor{gray}{rgb}{0.5,0.5,0.5}
\definecolor{mauve}{rgb}{0.58,0,0.82}

\lstset{frame=tb,
  language=MATLAB,
  aboveskip=3mm,
  belowskip=3mm,
  showstringspaces=false,
  columns=flexible,
  basicstyle={\small\ttfamily},
  numbers=none,
  numberstyle=\tiny\color{gray},
  keywordstyle=\color{blue},
  commentstyle=\color{dkgreen},
  stringstyle=\color{mauve},
  breaklines=false,
  breakatwhitespace=true,
  tabsize=3
}


%SetFonts

%SetFonts


\title{Scheduling Problem: How to Graduate}
\author{Julie Jiang}
%\date{}                            % Activate to display a given date or no date

\begin{document}
\maketitle
%\section{}
%\subsection{}
\section{Introduction}
College can be overwhelming, especially when it comes to trying to meet all the graduation requirements, which are quite interdependent. For example, an Applied Math major student a Tufts must complete 13 classes in the Math department. One of these classes is Math 87 Math Modeling, which requires the completion of Math 34 or 39 and Math 70 or 72. Additionally, Tufts undergraduates have to fulfill other distribution requirements in order to graduate, such as taking 6 semesters of a foreign language. A student's time is also limited -- one can only take 5 or 6 classes at most per semester, for a total of 8 semesters. \par
Given the constraints, what are the possible schedules of classes to take each semester so that a Math major can graduate on time? What about a Math and CS double major? What about a Math and CS double major who was placed out of all of her foreign language requirements?
\section{The Modeling Problem}
I will model the class scheduling problem for a typical student at Tufts with a precedence graph. Specifically, I will find the critical sequence of classes to graduation, and analyze the sensitivity and floats (if there are any) of every class. For example, a quick glance at the requirements for Math majors indicates that Calculus II is a very sensitive class to graduation -- but how sensitive? To illustrate my findings, I will build a Grantt chart. 

It is hard to find the ``best'' path to graduation, as this is a very subjective measure and cannot be well defined mathematically. However, I can perform a comparative analysis on the general level of difficulty of graduation of different concentrations of study. My model will also be able to measure how much harder, say, majoring in both Math and CS is compared to majoring in only Math.

This project should take me about 10 to 15 hours to complete. I plan to code in Python and build a small, interactive web app that can accept additional soft constraints and generate the path to graduation that maximizes these soft constraints. A soft constraint is one that is not neccessary but desirable, such as wanting to take only 4 classes every semester.

\section{Motivation}
Scheduling has always been a question of interest to me. What is the latest time I can get started on my Math 87 final project proposal to ensure that I deliver my best possible proposal? How do I maximize the utility of my time here at Tufts by taking all the classes that I want to take? Sparked by the intriguing lectures notes on scheduling problems on Trunk, I decided to fully investigate this issue of graduation. While this project only focuses on Tufts undergraduates, the same model can be generalized to be applicable for many other types of scheduling problems, as long as the constraints are well defined. 

\end{document}  